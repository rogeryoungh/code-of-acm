\documentclass{proart}

\title{ACM 数论专版}
\author{\href{https://github.com/rogeryoungh}{rogeryoungh}}

\begin{document}
\newcommand\mfrac[2]{\dfrac{#1\smash[b]{\strut}}{#2\smash[t]{\strut}}}
\newcommand{\RR}{\mathbb{R}}
\newcommand{\NN}{\mathbb{N}}
\newcommand{\QQ}{\mathbb{Q}}
\newcommand{\ZZ}{\mathbb{Z}}
\newcommand{\PP}{\mathbb{P}}
\newcommand{\ee}{\mathrm{e}}
\newcommand{\dd}{\mathrm{d}}
\newcommand{\uppi}{\mathrm{\pi}}
\newcommand{\eps}{\varepsilon}

\newcommand{\lcm}{\operatorname{lcm}}
\newcommand{\DFT}{\operatorname{DFT}_{\omega_n}}
\newcommand{\Id}{\operatorname{Id}}

\newcommand{\num}[1]{{\fzfs{(}}{\rm{#1}}{\fzfs{)}}}

\maketitle

\tableofcontents

\begin{abstract}
    在代码模板里插入太多公式,翻看反而不方便,不如单独开一章。
\end{abstract}

\section{数论函数}

数论函数是 $\mathbb{N}^+ \to \mathbb{C}$ 的函数。

\paragraph{定义}
常见函数 $\varepsilon(n) \coloneqq [n=1]$,$1(n) \coloneqq 1$,$\Id_k(n) \coloneqq n^k$,$\sigma_k(n) = \sum\limits_{d \mid n} d^k$。

设 $n > 1$ 的唯一分解式是
\[ n = p_1^{k_1}p_2^{k_2}\cdots p_s^{k_s} \]
则 $\omega(n) = s$,$\Omega(n) = \sum k_i$。$\nu(n) = (-1)^{\omega(n)}$,$\lambda(n) = (-1)^{\Omega(n)}$。

Euler $\varphi$ 函数:
\[ \varphi(m) \coloneqq \sum_{i=1}^m [\gcd(i,m) = 1] = m \prod_{p \mid n}\left( 1-\frac{1}{p} \right) \]

Mobius $\mu$ 函数:
\[ \mu(n) = \begin{cases}
    1, &n=1\\
    0, &\exists d > 1, d^2 \mid n\\
    (-1)^{\omega(n)}, &\text{otherwise}
\end{cases} \]

若数论函数 $f$ 满足对 $\gcd(a,b)$ 有 $f(ab) = f(a)f(b)$,则称为积性函数;若有 $f(ab) = f(a) + f(b)$ 则称为加性函数。若 $f,g$ 为积性函数,则 $f(x^p),f^p(x),fg(x),f \ast g$ 都是积性函数。

积性函数有 $\varphi,\varepsilon,1,\Id_k,\sigma_k,\nu,\lambda$,加性函数有 $\omega,\Omega$。

\paragraph{性质}
$\varphi(m)$ 的性质:$\varphi(p^k) = (p-1)p^{k-1}$。

\[ \varphi(m) = m \prod_{p \mid n}\left( 1-\frac{1}{p} \right), \sum_{d \mid m} \varphi(d) = m \]

$\mu(m)$ 的性质:$\sum\limits_{d \mid n} \mu(d) = [n=1] = \varepsilon$。

Femmat - Euler 定理:当 $\gcd(a,m) = 1$ 时,有
\[ a^{\varphi(m)} \equiv 1 \pmod m \]
推广,不要求互质,当 $b \geqslant \varphi(m)$ 时有
\[ a^b \equiv a^{b \bmod \varphi(m) + \varphi(m) } \]


\section{生成函数}

\paragraph{普通生成函数}
可以把序列 $\{a_n\}$ 映射到形式幂级数
\[ a(x) = a_0 + a_1 x + a_2 x^2 + \cdots + a_nx^n + \cdots \]
这样就可以通过对级数的研究得到序列的性质。记作 $f_n = [x^n]F(x)$。

生成函数的四则运算是显然的。其中 $F(x)G(x) = \{\sum\limits_{i=0}^n f_ig_{n-i}\}$ 为序列 $\{f_n\}$ 与 $\{g_n\}$ 的卷积。

对生成函数移位可以乘除 $x^m$,也可以对生成函数逐项求导、逐项积分。求生成函数的部分和,即是 $(1-x)^{-1}F(x)$。

基础函数:
\begin{equation*}
    \begin{aligned}
        \frac{1}{1-cx} &= \sum_{k=0}^\infty c^nx^n\\
        \frac{x}{1-x-x^2} &= \sum_{k=0}^\infty F_nx^n \\
        \ee^{cx} &= \sum_{k=0}^\infty \frac{c^nx^n}{n!} \\
        \ln(1+x) &= \sum_{k=1}^\infty \frac{(-1)^{n-1}}{n} x^n\\
        -\ln(1-x) &= \sum_{k=1}^\infty \frac{1}{n} x^n
    \end{aligned}
\end{equation*}

五边形数定理:
\[ \prod_{i=1}^\infty (1-x^i) = \sum_{k=0}^{\infty} (-1)^k x^{k(3k \pm 1) /2} \]



\section{容斥原理}

对于有限集 $S$,它的子集 $A,B,C$ 有
\[ |A \cup B \cup C| = |A| + |B| + |C| - |A\cap B| - |B \cap C| - |C \cap A| + |A \cap B \cap C| \]
对于 $n$ 个集合的情况,

容斥原理:设有限集 $S$ 的 $n$ 个子集 $A_i$,则有
\[ \left|\bigcup _{i=1}^{n}A_{i}\right|=\sum _{\varnothing \neq J\subseteq \{1,2,\ldots ,n\}}(-1)^{|J|-1}\left| \bigcap _{j\in J}A_{j} \right| \]

\paragraph{Burnside 引理}
设 $A$ 和 $B$ 为有限集合,$X=B^A$ 表示所有从 $A \to B$ 的映射。$G$ 是 $A$ 上的置换群,$X/G$ 表示 $G$ 作用在 $X$ 上产生的所有等价类的集合(若 $X$ 中的两个映射经过 $G$ 中的置换作用后相等,则它们在同一等价类中),则
\[ |X/G|=\frac{1}{|G|}\sum_{g\in G}|X^g| \]
其中 $|S|$ 表示集合 $S$ 中元素的个数,且
\[ X^g = \{ x \mid x \in X,g(x) = x \} \]

\paragraph{Polya 定理}
前置条件与 Burnside 引理相同
\[ |X/G|=\frac{1}{|G|}\sum_{g\in G}|B|^{c(g)} \]
其中 $c(g)$ 表示置换 $g$ 能拆分成不相交的循环置换的数量。


\section{Mobius 反演}

对于数论函数 $f(x)$ 和 $g(x)$,定义 $h(x)$ 为
\[ h(x) = \sum_{d \mid x} f(d) g\left(\frac{x}{d}\right) = \sum_{ab = x} f(a)g(b) \]
为其 Dirichlet 卷积,简记为 $h = f \ast g$。

Dirichlet 卷积满足交换律、结合律、分配律等,其上有单位元 $\varepsilon$ 和逆元,换言之其与数论函数构成了一个整环。

对于数论函数 $f(n)$,有 $F = f \ast 1 \Rightarrow f = F \ast \mu$。

即对于数论函数 $f(n)$,有
\[ F(n) = \sum_{d \mid n} f(d) \Rightarrow f(n) = \sum_{d \mid n} \mu(d)f\left(\frac{n}{d}\right) \]
以及
\[ F(n) = \sum_{n \mid d} f(d) \Rightarrow f(n) = \sum_{n \mid d} \mu\left(\frac{d}{n}\right)F(d) \]
常见的有
\[ \begin{matrix}
    1 = \varepsilon \ast 1, & \sigma_k = \Id_k \ast 1,& d = 1 \ast 1 \\
    \Id = \varphi \ast 1, & 1_{\operatorname{Sq}} = \lambda \ast 1, &(d \ast 1)^2 = d^3 \ast 1
\end{matrix} \]

\section{组和排列}

\subsection{数学常数}

\paragraph{排列数}
从 $n$ 个不同元素中取 $m$ 个不同元素按顺序排成的一列称作一个排列,则排列种数为 
\[ A_n^m = n(n-1)\cdots (n-m+1) = \frac{n!}{(n-m)!} \]
若元素可重复取出,则为 $n^m$。

\paragraph{组和数}
从 $n$ 个不同元素中取 $m$ 个不同元素组成的一组称作一个组合,则组合种数为
\[ C_n^m = \frac{n(n-1)\cdots(n-m+1)}{m!} = \frac{n!}{m!(n-m)!} \]
也记作 $\binom{n}{m}$。若元素可重复取出,则为 $\binom{n+m-1}{m}$。

\paragraph{圆排列}
将 $n$ 个元素不分首尾的排成一圈称作一个圆排列,则圆排列种数为 $(n-1)!$。

\paragraph{隔板法}
即 $n$ 个相同的球放进 $k$ 个盒子(要求盒子非空)里的种数为 $\binom{n-1}{m-1}$,即不定方程 $\sum x_m = n$ 的正整数解数。若允许盒子为空,则为 $\binom{n+k-1}{k-1}$。

\paragraph{错排公式}
每一个元素都不在自己的位置上的 $n$ 元排列称为一个错排,则错排种数为
\[ D_n = n! \left( 1 - \frac{1}{1!} + \frac{1}{2!} - \frac{1}{3!} + \cdots + \frac{(-1)^n}{n!} \right) = \left\lfloor \frac{n!}{\ee} + 0.5 \right\rfloor \]
其递推式为 $D_n = (n-1)(D_{n-1} + D_{n-2})$。

\subsection{常见公式}

\paragraph{组和数和式}
\[ \binom{n+1}{k} = \binom{n}{k} + \binom{n}{k-1}, \quad \binom{n}{k} = \frac{n}{k} \binom{n-1}{k-1} \]
\[ \sum\limits_{r=0}^k \binom{n+r-1}{r} = \binom{n+k}{k}, \quad \sum\limits_{r=0}^n \binom{n}{r}^2 = \binom{2n}{n} \]
\[ \binom{n}{i}\binom{i}{m} = \binom{n}{m}\binom{n-m}{i-m}, \quad \sum\limits_{i=m}^n \binom{i}{a} = \binom{n+1}{a+1} - \binom{m}{a+1} \]
\[ \sum\limits_{i=0}^k \binom{n}{i}\binom{m}{k-i} = \binom{n+m}{k}, \quad \sum\limits_{j=0}^k \binom{k}{j}^2 \binom{n+2k-j}{2k} = \binom{n+k}{k}^2 \]

\[ \sum\limits_{r=0}^n \binom{dn}{dr} = \frac{1}{d} \sum_{r=1}^d  \left( 1+\ee^{\tfrac{2\pi r i}{d}} \right)^{dn} \]

\paragraph{二项式反演}

注意到
\[ \sum_{k=0}^n (-1)^k \binom{n}{k} = [n=0] \]
于是有
\[ f(n) = \sum_{k=0}^n \binom{n}{k} g(k) \Rightarrow g(n) = \sum_{k=0}^n (-1)^{n-k}\binom{n}{k} f(k) \]

\paragraph{斯特林数}

第一类斯特林数$\begin{bmatrix}n\\ m\end{bmatrix}$表示将$n$个不同元素构成$m$个圆排列的数目。

第二类斯特林数 $\begin{Bmatrix} n \\m \end{Bmatrix}$ 表示把 $n$ 个不同元素划分成$m$个相同的集合(不能有空集)的方案数。

\[
m^n=\sum_{i=0}^m \binom m i  \left\{ \begin{matrix} n \\ i \end{matrix} \right\} i!
\]
\[\left\{ \begin{matrix} n\\m \end{matrix} \right\}=\sum_{i=0}^m \frac{i^n}{i!} \frac{(-1)^{m-i}}{(m-i)!}\]

\[\begin{bmatrix}n \\ m\end{bmatrix}=\begin{bmatrix}n-1 \\ m-1\end{bmatrix}+(n-1) \begin{bmatrix}n-1 \\ m\end{bmatrix}\]

\[\begin{Bmatrix}n \\ m\end{Bmatrix}=\begin{Bmatrix}n-1 \\ m-1\end{Bmatrix}+m \begin{Bmatrix}n-1 \\ m\end{Bmatrix}\]

一行第二类斯特林数的和是贝尔数。

\end{document}
