\documentclass{probook}

\let\problem\relax

\DeclareTColorBox[no counter]{problem}{ o t\label g }{
    claim,
    IfValueTF={#1}{title={题目大意$\quad$}}{title=题目大意$\quad$},
    IfBooleanTF={#2}{label=#3}{}}

\title{ACM 题解}
\author{\href{https://github.com/rogeryoungh}{rogeryoungh}}

\begin{document}
\newcommand\mfrac[2]{\dfrac{#1\smash[b]{\strut}}{#2\smash[t]{\strut}}}
\newcommand{\RR}{\mathbb{R}}
\newcommand{\NN}{\mathbb{N}}
\newcommand{\QQ}{\mathbb{Q}}
\newcommand{\ZZ}{\mathbb{Z}}
\newcommand{\ee}{\mathrm{e}}
\newcommand{\dd}{\mathrm{d}}
\newcommand{\uppi}{\mathrm{\pi}}
\newcommand{\eps}{\varepsilon}

\newcommand{\num}[1]{{\fzfs{(}}{\rm{#1}}{\fzfs{)}}}

\maketitle

\frontmatter

\chapter*{前言}

这里是我的 ACM 训练日志兼题解。

并不在题目中放全部代码,请在 archive 文件夹中寻找对应代码。

\tableofcontents

\mainmatter

\chapter{2020 年 9 月}

\section{P1004 方格取数}

\paragraph{题目大意}

在 $n \times n$ 的方格($n \leqslant 9$)中存在一些正整数,经过格子时获得格子上的数,但只能获得一次。

某人只能向右或向下走,从格子的左上角走到到右下角,共走两次,求最大能取得的数字。

\paragraph{分析}

考虑把先走后走转化为两个人同时走,只需要处理遇到两次的值即可。

考虑四维 DP,用 $dp[x_1,y_1,x_2,y_2]$ 表示第一个人走到 $(x_1,y_1)$ 和第二个人走到 $(x_2,y_2)$。再考虑转移,每一个位置仅可能从其左面或上面转移来,于是当 $x_1 \ne y_1$ 或 $x_2 \ne y_2$ 时有
\[ dp[x_1,y_1,x_2,y_2] = \max\left\{ \begin{matrix}
dp[x_1-1,y_1,x_2-1,y_2] \\
dp[x_1,y_1-1,x_2-1,y_2] \\
dp[x_1-1,y_1,x_2,y_2-1] \\
dp[x_1,y_1-1,x_2,y_2-1]
\end{matrix}\right\}  + a[x_1,y_1] + a[x_2,y_2] \]
当 $x_1=y_1,x_2=y_2$ 时,只加一次即可。到这里 $O(n^4)$ 其实已经可以过题了,但还可以优化。

注意到一些状态是不可达的,因为 $x_1+y_1 = x_2+y_2$,因此存在 $O(n^3)$ 的 DP。考虑当前已走长度 $h=x_1+y_1=x_2+y_2$,于是可以把两个人的座标表示为 $(x_1,h-x_1)$ 和 $(x_2,h-x_2)$。于是记状态为 $dp[h,x_1,x_2]$,当 $x_1 \ne x_2$ 时有
\[ dp[h,x_1,x_2] = \max\left\{ \begin{matrix}
dp[h-1,x_1,x_2] \\
dp[h-1,x_1,x_2-1] \\
dp[h-1,x_1-1,x_2] \\
dp[h-1,x_1-1,x_2-1]
\end{matrix}\right\}  + a[x_1,h-x_1] + a[x_2,h-x_2] \]
再注意到可以使用滚动数组,最后写的时候注意上下标的范围即可。

\section{CF1384B2 Koa and the Beach}

\paragraph{题目大意}

海里每一个位置都有一个深度,而且水里随着时间有锯齿状周期为 $2k$ 的潮汐。当潮汐与深度之和大于给定值 $l$ 时 Koa 会溺水,游泳、海岸、岛上永远是安全的。

在任意时间 Koa 可以选择游到 $x+1$ 或停留在 $x$,试问 Koa 是否能够安全的从海岸 $0$ 到达岛上 $n+1$。

\paragraph{分析}

开始的想法是随着时间 DP,更好的解法是贪心。

若一个位置在水位最高时仍不会溺水,则称这个位置是安全的,并且可以任意选择出发时机。即安全位置之间是相互独立的,我们只需判断可达性,尽量到达每一个安全位置。若之后 $2k$ 个位置没有安全位置,必死。

Koa 的最优决策是在刚退潮时出发,如果能走就尽量往前走,不能走就等,等到涨潮就说明不存在通过方法。

\section{P2181 对角线}

\paragraph{题目大意}

凸 $n$ 边形中,任意三条对角线不共点,求所有对角线交点的个数。

\paragraph{分析}

注意到一个交点对应凸多边形 $4$ 个定点,于是等价于 $n$ 个点任选 $4$ 个点的选法种数,即
\[
\binom{n}{4} = \frac{n(n-1)(n-2)(n-3)}{24}
\]
注意爆 \verb`long long`,需要写成 \verb`n * (n-1) / 2 * (n-2) / 3 * (n-3) / 4`。

\chapter{2020 年 10 月}

\section{P1095 守望者的逃离}

\paragraph{题目大意}

守望者的可以在一秒逃出 $17 {\rm m}$,或者消耗 $10$ 点魔法值闪现 $60 {\rm m}$。原地休息时每秒回复 $4$ 点魔法值。

守望者开始有 $M$ 点魔法,需要在 $T$ 秒内逃离距离 $S$。若能逃离则求最短逃离时间,否则求最远距离。

\paragraph{分析}

设 $s_1$为一直走路,$s_2$ 为一直闪现恢复。当 $s_2$ 快了就把 $s_1$ 更新为 $s_2$。

\section{P1923 求第 $k$ 小的数}

\paragraph{题目大意}

给定数列,求第 $k$ 小的数。

\paragraph{分析}

考虑分治,随便选一个数 $x$,然后把比 $x$ 大的数移到右边,比 $x$ 小的数移到左边。因此得到了数 $x$ 的排位,若恰为第 $k$ 个则返回,否则根据大小在左右寻找。

该算法是不稳定的,期望复杂度 $O(n)$,最坏复杂度 $O(n^2)$。实际上存在最坏复杂度为 $O(n)$ 的 BFPTR 算法,但因为其常数过大实现复杂而很少应用。

\section{P1928 外星密码}

\paragraph{题目大意}

我们将重复的 $n$ 个字符串 \verb`S` 压缩为 \verb`[nS]`,且存在多重压缩。给定一个压缩结果,展开它。

\paragraph{分析}

考虑用类似状态机的方式解析字符串。当读到正常字符时,把它添到 $s$ 末尾;读到左括号时,则递归 \verb`read()`,重复 $n$ 遍添到 $s$ 末尾;右括号或文本结束则返回 $s$。

\section{P1990 覆盖墙壁}

\paragraph{题目大意}

有 \verb`I` 形和 \verb`L` 形两种砖头,分别能覆盖 2 个单元和 3 个单元。求 $2 \times n$ 的墙有多少不重复的覆盖方式,结果对 $10^4$ 取模。

\paragraph{分析}

其中 \verb`I` 形砖块仅有横放和竖放两种。关键在于 \verb`L` 形,两个 \verb`L` 形之间可以用 \verb`I` 形填充,这让情况变得复杂起来。

对于 $F_n$ 的递推,我们可以想到:在 $F_{n-1}$ 后放一个 \verb`I` 形砖块;在 $F_{n-2}$ 后放两个横着的 \verb`I` 形砖块。对于更前面的递推,较为复杂。

两个 \verb`L` 形砖块对齐,上下翻转也可以,即 $2 F_{n-3}$;两个 \verb`L` 形砖块可以对顶放,空缺恰用一个 \verb`I` 填充,即 $2 F_{n-4}$;类似 $2F_{n-3}$,中间可以再插入两个 \verb`I` 形,即 $2 F_{n-5}$……直到 \verb`I` 形和 \verb`L` 形砖块恰好铺满墙壁,即 $2F_{0}$。

容易得到我们的递推式
\[ F_n = F_{n-1} + F_{n-2} + 2 \sum_{i=0}^{n-3} F_i \]
利用错位相减法,不难化简得到
\[ F_n = 2 F_{n-1} + F_{n-3} \]

\section{P1090 合并果子}

\paragraph{题目大意}

可以合并两堆果子成一堆新果子,消耗两堆果子数目之和的体力。给定 $n$ 堆果子的数目 $a_i$,求体力耗费最小的方案。

\paragraph{分析}

很容易猜到贪心结论,不断选取两个最小的堆进行合并。本质上是证明 Huffman 树的构造的正确性,有点复杂。

\section{P4995 跳跳!}

\paragraph{题目大意}

给定石头的高度 $h$,从第 $i$ 块石头跳到第 $j$ 块上耗费体力 $(h_i-h_j)^2$ ,求最耗体力的路径。

\paragraph{分析}

容易猜到贪心结论,是不断的在剩余石头中最大最小的来回跳。

考虑证明结论,设 $h_i$ 是将要跳的序列,展开求和式
\[ S = \sum_{k=1}^{n-1}(h_k-h_{k+1})^2 = \sum_{k=1}^nh_k^2 - 2\sum_{k=1}^{n-1}h_kh_{k+1} \]
注意到平方和为一个定值,重点在后半式。记 $H_k = h_{k+1}$,有
\[ \sum_{k=1}^{n-1}h_kH_k \]
利用高中时学的排序不等式,有
\[ \text{反序和} \leqslant \text{乱序和} \leqslant \text{顺序和} \]
于是有反序最小。

\section{P1077 摆花}

\paragraph{题目大意}

要从 $n$ 种花中挑出 $m$ 盆展览,其中第 $i$ 种花不得多于 $a_i$ 种。求有几种选法。

\paragraph{分析}

考虑动态规划,记状态 $dp[i,j]$ 为摆完前 $i$ 种花,共 $j$ 盆时的方案数。容易得到递推式
\[ dp[i,j] = \sum_{k = j - \min(a_i, j)}^j dp[i-1,k] \]
边界条件是 $dp[0,0] = 1$。可以用滚动数组、前缀和优化。

\chapter{2020 年 11 月}

\section{P1880 石子合并}

\paragraph{题目大意}

环形队列上有 $n$ 堆石子,可以把相邻的两堆合成一堆,每次合并的得分是新一堆的石子数。求最终分数的最小值和最大值。

\paragraph{分析}

考虑 $dp(i,j)$ 是将区间 $[i,j]$ 的石子全部合并的最大值。于是状态转移方程为
\[ dp(i,j) = \max_{i \leqslant k \leqslant j}(dp(i,k) + dp(k+1,j) + s(i,j)) \]
其中 $s(i,j)$ 是 $[i,j]$ 中所有石子数。

然而不能通过先 $i$ 再 $j$ 再 $k$ 的循环来递推,运算顺序值得注意。

\chapter{2021 年 3月}

\section{CF1494C 1D Sokoban}

\paragraph{题目大意}

假设一个直线上的推箱子游戏,你的出生点位于 $0$,在 $a_i$ 处有 $n$ 个箱子,在 $b_i$ 处有 $m$ 个目标点。箱子可能初始化在目标点,但不会在 $0$。

就像推箱子一样,你可以把箱子推到目标点而不能越过箱子。求最多能使目标点上有几个箱子。

\paragraph{分析}

首先看正半轴,推的时候箱子会积起来,关注点在“箱子队列”的右端。

容易发现,当右端未碰到新目标点时,结果是不可能变多的。分类讨论

1. 当右端碰到恰在目标点的新箱子时,\verb`ans+1`,计算此时覆盖个数,取最值,\verb`len+1`。

2. 当右端碰到箱子时,\verb`len+1`。

3. 当右端碰到目标点时,计算此时覆盖个数,取最值。

计算箱子下覆盖目标点个数可以用双指针法维护。尝试了一下二分,但好像没有变快啊

\section{P1019 单词接龙}

\paragraph{题目大意}

对于字符串 $a,b$,若 $a$ 的末尾与 $b$ 的开头有部分字符串相同,则其可以拼接起来。例如 \verb`at + tact = atact`。

给定词典和初始字母,每个单词最多出现两次,求最大拼接长度。

\paragraph{分析}

我是没想出来,经题解提示了拼接函数才写出来的。

设 $mt(x,y)$ 为第 $x$ 和 $y$ 个单词拼接后的最小重合长度,其可简单的通过匹配得到。

然后回溯 dfs,搜索即可。

\section{P2678 跳石头}

\paragraph{题目大意}

在 $0$ 到 $L$ 的位置间,有 $N$ 块岩石。若要使得岩石间距最小值最大,允许移除 $M$ 块,求此时间距最小值。

\paragraph{分析}

定义函数 $f(x)$,输入为最小值,输出为被移除岩石的个数,实现如下

显然 $f(x)$ 是单调递减的函数,二分查找即可。

\chapter{2021 年 4 月}

\section{P1314 聪明的质检员}

\paragraph{题目大意}

对于一个区间 $[l_i,r_i]$,计算矿石在这个区间上的检验值 $y_i$:
\[ y_i=\sum_{j=l_i}^{r_i}[w_j \geqslant W] \times \sum_{j=l_i}^{r_i}[w_j \geqslant W]v_j \]
记检验结果为 $y=\sum y_i$,对于给定的 $s$,求 $|y-s|$ 的最小值。

\paragraph{分析}

注意到 $y(w)$ 是关于 $W$ 的递减函数,对 $w$ 在 $[w_{\min},w_{\max}]$ 上进行二分。

区间求和需要用前缀和优化。

\section{P2261 余数求和}

\paragraph{题目大意}

给出正整数 $n$ 和 $k$,请计算
\[ G(n, k) = \sum_{i = 1}^n k \bmod i \]

\paragraph{分析}

因为
\[ k \bmod i = k - i \left\lfloor \frac{k}{i} \right\rfloor \]
因此有
\[ G(n,k) = nk - \sum_{i=1}^n i \left\lfloor \frac{k}{i} \right\rfloor  \]
后面需要整数分块。

\chapter{2021 年 5 月}

\section{P2522 Problem B}

\paragraph{题目大意}

即求
\[ \sum_{i=a}^b \sum_{j=c}^d [\gcd(i,j) = k] \]

\paragraph{分析}

容易想到,独立出函数 $f(k)$ 使得
\[ f(k) = \sum_{i=1}^x \sum_{j=1}^y [\gcd(i,j) = k] \]

利用 Mobius 反演化简,设 $F(d)$
\[ F(n) = \sum_{n \mid d} f(d) = \sum_{i=1}^x \sum_{j=1}^y [n \mid i][n \mid j] = \left\lfloor \frac{x}{n} \right\rfloor \left\lfloor \frac{y}{n} \right\rfloor \]

反演化简有
\[ f(n) = \sum_{n \mid d} \mu\left(\frac{d}{n}\right)F(d) = \sum_{t=1}^{\min(x,y)} \mu(t) \left\lfloor \frac{x}{tn} \right\rfloor \left\lfloor \frac{y}{tn} \right\rfloor \]

预处理出 $\mu(t)$ 的前缀和,剩下的就是一个二重分块了。

\paragraph{类似题目}

P2158 仪仗队:即 $k = 1$ 的特殊情况。

P3455 ZAP-Queries:几乎一样。

P2257 YY 的 GCD:比这题难,单独开篇。

\section{P2257 YY 的 GCD}

\paragraph{题目大意}
\[ \sum_{i=1}^N \sum_{j=1}^M [\gcd(i,j) \in \mathbb{P}] \]

\paragraph{分析}

先转化一下
\[ \sum_{i=1}^N \sum_{j=1}^M [\gcd(i,j) \in \mathbb{P}] = \sum_{p}\sum_{i=1}^N \sum_{j=1}^M [\gcd(i,j) = p] \]
在 P2522 中得到
\[ \sum_{i=1}^x \sum_{j=1}^y [\gcd(i,j) = k] = \sum_{t=1}^{\min(x,y)} \mu(t) \left\lfloor \frac{x}{tk} \right\rfloor \left\lfloor \frac{y}{tk} \right\rfloor \]
代入有
\[ \sum_{p} \sum_{t=1}^{\min(x,y)} \mu(t) \left\lfloor \frac{x}{tp} \right\rfloor \left\lfloor \frac{y}{tp} \right\rfloor \]
令 $T = tp$,$T$ 的上界应还是 $\min(x,y)$,代入有
\[ \sum_{T=1}^{\min(x,y)}  \left\lfloor \frac{x}{T} \right\rfloor \left\lfloor \frac{y}{T} \right\rfloor \sum_{p \mid T} \mu\left(\frac{T}{p}\right) \]
后面部分可以利用 Euler 筛预处理的。令
\[ dp(T) = \sum_{p \mid T} \mu\left(\frac{T}{p}\right) \]
$f_i$ 是其前缀和,最后整除分块即可。

\section{P2303 Longge 的问题}

\paragraph{题目大意}

即求
\[ \sum_{i=1}^n \gcd(i,n) \]

\paragraph{分析}

联想到
\[ \varphi(n) = \sum_{i=1}^n [\gcd(i,n) = 1] \]
尝试凑这个形式
\[ \begin{aligned}
\sum_{i=1}^n \gcd(i,n) &= \sum_{d \mid n} d \sum_{i=1}^n [\gcd(i,n) = d] \\
&= \sum_{d \mid n} d \sum_{i=1}^{n/d} [\gcd(i,n/d) = 1] \\
&= \sum_{d \mid n} d \varphi(n/d)
\end{aligned} \]
这里其实已经可以过题了,但还可以再瞎搞一下,令
\[ \sum_{d \mid n} d \varphi(n/d) = n\sum_{d \mid n} \frac{\varphi(d)}{d} = nf(n)\]
尝试证明 $f(n)$ 是一个积性函数。设 $\gcd(a,b) = 1$,有
\[ \begin{aligned}
f(a)f(b) &= \left(\sum_{d_1 \mid a} \frac{\varphi(d_1)}{d_1}\right) \left(\sum_{d_2 \mid b} \frac{\varphi(d_2)}{d_2}\right)\\
&= \sum_{d_1 \mid a} \sum_{d_2 \mid b} \frac{\varphi(d_1)}{d_1} \frac{\varphi(d_2)}{d_2}\\
&= \sum_{d_1 \mid a} \sum_{d_2 \mid b} \frac{\varphi(d_1d_2)}{d_1d_2}\\
&= f(ab)
\end{aligned} \]
再来推一下素数,注意 $1 \mid p^k$,有
\[ f(p^k) = \sum_{d \mid p^k} \frac{\varphi(d)}{d} = \sum_{i=0}^k \frac{\varphi(p^i)}{p^i} = k\left(1 - \frac{1}{p}\right) + 1 \]
类似于 $\varphi(m)$ 唯一分解形式,我们还有
\[ f(n) = \prod_{i=1}^sf(p_i^{k_i}) = \prod_{i=1}^s \frac{k_ip_i - k_i + p_i}{p_i} \]

\section{P2350 外星人}

\paragraph{题目大意}

设
\[ \varphi_{x+1}(m) = \varphi(\varphi_x(m)) \]
求最小的 $x$ 使得 $\varphi_x(m) = 1$。

\paragraph{分析}

注意到仅有 $\varphi(1) = \varphi(2) = 1$,再有公式
\[ \varphi\left(\prod_{i=1}^mp_i^{q_i}\right) = \prod_{i=1}^m(p_i-1)p_i^{q_i-1} \]
因此从唯一分解形式的角度来看,迭代一次消去了一个 $2$,也生成了一些 $2$。

考虑设 $f(n)$ 为 $\varphi(n)$ 中因子 $2$ 的个数。设 $\gcd(a,b) = 1$,可以证明 $f(ab) = f(a) f(b)$。同时 $f(p) = f(p-1)$。这说明 $f(x)$ 是一个积性函数,可以用 Euler 筛递推。

注意若没有质因子 $2$,则答案需要加 $1$。
\chapter{2021 年 7 月}

\section{P1762 偶数}

\paragraph{题目大意}

求杨辉三角形前 $n$ 行的偶数个数,对 $1000003$ 取模。

\paragraph{分析}

对杨辉三角奇数打表。于是递推公式
\[ f(2^t+n) = f(2^t) + 2f(n) \]
就是显然的了,那么偶数即是全部的减去奇数个数。



\chapter{2021 年 8 月}

\section{LOJ6053 简单的函数}

\paragraph{题目大意}

给定积性函数 $f(p^c) = p \oplus c$。对 $1000000007$ 取模求前缀和。

\paragraph{分析}

Min25 板子题。

\chapter{2021 年 9 月}

\section{HDU6537 Neko and function}

\paragraph{题目大意}

定义 $f(n, k)$ 为满足如下要求的,长度为 $k$ 的数列 $\\{a_i\\}$ 个数:

- 要求每个数字都大于 $1$;
- 数列各项之积恰为 $n$。

求其前缀和 $S(n, k) = \sum f(i, k)$。

\paragraph{分析}

多组输入好坑啊。

先允许 $a_i=1$,将结果记为 $g(n,k)$,容易利用隔板法求得
\[ g (p^c, k) = \binom{t + k - 1}{k - 1} \]
对于多个质因数,显然其贡献是相互独立的,即 $g$ 是积性函数。考虑选取 $i$ 个数令其 $a_i>1$,枚举有
\[ g (n, k) = \sum_{i = 1}^k \binom{n}{i} f (n, i) \]
二项式反演(或者直接容斥)有
\[ f (n, k) = \sum_{i = 1}^k (- 1)^{k - i} \binom{k}{i} g (n, i) \]
求和并换序
\[ \begin{aligned}
\sum_{i = 1}^x f (i, k) &= \sum_{n = 1}^x \sum_{i = 1}^k (- 1)^{k - i} \binom{k}{i} g (n, i) \\
&= \sum_{i = 1}^k (- 1)^{k - i} \binom{k}{i} \sum_{n = 1}^x g (n, i)
\end{aligned} \]
后面即是 $g$ 的前缀和,可以用 Min25 筛。

\section{HDU6750 Function}

\paragraph{题目大意}

定义
\[ f (n) = \sum_{d \mid n} d \left[ \gcd \left( d, \frac{n}{d} \right) = 1 \right] \]
求其前缀和 $S(n)$。

\paragraph{分析}

首先 Min25 筛是过不了的,内存不够。先推式子
\[ \begin{aligned}
S (n) &= \sum_{i = 1}^n \sum_{d \mid i} d \left[ \gcd \left( d, \frac{i}{d} \right) = 1 \right] \\
&= \sum_{d = 1}^n \sum_{i = 1}^n d [d \mid i] \left[\gcd \left( d, \frac{i}{d} \right) = 1 \right] \\
&= \sum_{d = 1}^n \sum_{i = 1}^{n / d} d [\gcd (i, d) = 1]
\end{aligned} \]
看到 $\gcd = 1$ 就应该想起来 Mobius 反演,容易化简有
\[ S (n) = \sum_{d = 1}^n \sum_{i = 1}^{n / d} d \sum_{u \mid \gcd (i, d)} \mu(u) = \sum_{u = 1}^{\sqrt{n}} u \mu (u) \sum_{d = 1}^{n / u^2} d \left\lfloor \frac{n}{d u^2} \right\rfloor \]
用数论分块即可,计算复杂度 $O ( \sqrt{n} \log n )$。

\section{HDU6755 Fibonacci Sum}

\paragraph{题目大意}

设 Fibonacci 数列第 $i$ 项为 $F_i$,求
\[ S = \sum_{i = 0}^n (F_{i c})^k \]
\paragraph{分析}

由特征方程法,设
\[ x^2 + x - 1 = 0 \Longrightarrow A, B = \frac{1 \pm \sqrt{5}}{2} \]
因此 Fibonacci 通项公式即可表示为
\[ F_n = \frac{A^n - B^n}{A - B} \]
根据二次剩余的知识,模意义下是可以开方的。因此
\[ (F_{i c})^k = \left( \frac{A^{i c} - B^{i c}}{A - B} \right)^k
= \frac{1}{(A - B)^k} \sum_{j = 0}^k \binom{k}{j} (- 1)^{k - j} (A^j B^{k - j})^{i c} \]
求和有
\[ (A - B)^k S = (A - B)^k \sum_{i = 0}^n (F_{i c})^k = \sum_{i = 0}^n \sum_{j
= 0}^k \binom{k}{j} (- 1)^{k - j} (A^j B^{k - j})^{i c} \]
换序求和,即是等比数列(公比可能为 $1$,需要特判)
\[ \begin{aligned}
(A-B)^{k}S
& =\sum_{j=0}^{k}\binom{k}{j}(-1)^{k-j}\sum_{i=0}^{n}(A^{j}B^{k-j})^{ic}\\
& =\sum_{j=0}^{k}\binom{k}{j}(-1)^{k-j}{\frac{(A^{j}B^{k-j})^{c(n+1)}-1}{(A^{j}B^{k-j})^{c}-1}}
\end{aligned} \]
直接计算会 TLE,需要用中间变量简化
\[ \frac{(A^j B^{k - j})^{c (n + 1)} - 1}{(A^j B^{k - j})^c - 1} = \frac{B^{k c (n + 1)} {(A B^{- 1})^{c (n + 1) j}}  - 1}{B^{k c} (A B^{- 1})^{c j} - 1} \]
用中间变量在循环中递推,再加上 Euler 降幂,就可以过题了。

\section{HDU6833 A Very Easy Math Problem}

\paragraph{题目大意}

即求
\[ S (n) = \sum^n_{\{ a_x \} = 1} \left( \prod_{j = 1}^x a_j \right)^k f (\gcd \{ a_x \}) \gcd \{ a_x \} \]
我自己随便简写了,全打太麻烦了。

\paragraph{分析}

这东西看着很吓人,其实是纸老虎。首先 $f(x) = \mu(x)^2$。提出 $\gcd$ 有
\[ \begin{aligned}
S (n) & =\sum_{d=1}^{n}\sum^{n}_{\{ a_{x}\}=1}(\prod_{j=1}^{x}a_{j})^{k}{\mu}(d)^{2}d [gcd \{ a_{x}\} =d]\\
& =\sum_{d=1}^{n}{\mu}(d)^{2}d^{k x+1}\sum^{n/d}_{\{ a_{x}\} =1}(\prod_{j=1}^{x}a_{j})^{k}[gcd \{ a_{x}\} =1]
\end{aligned} \]
对后面这部分反演有
\[ \begin{aligned}
& \sum^{n / d}_{\{ a_x \} = 1} \left( \prod_{j = 1}^x a_j \right)^k \left( \sum_{t \mid \gcd \{ a_x \}} \mu (t) \right)\\
= & \sum^{n / d}_{\{ a_x \} = 1} \left( \prod_{j = 1}^x a_j \right)^k \left( \sum_{t = 1}^{n / d} [t \mid \gcd \{ a_x \}] \mu (t) \right)\\
= & \sum_{t = 1}^{n / d} \mu (t) t^{k x} \sum^{n / d t}_{\{ a_x \} = 1} \left( \prod_{j = 1}^x a_j \right)^k
\end{aligned} \]
代回去,枚举 $T = d t$
\[ \begin{aligned}
S (n) & = \sum_{d = 1}^n d \mu (d)^2 (d t)^{k x} \sum_{t = 1}^{n / d} \mu (t) \sum^{n / d t}_{\{ a_x \} = 1} \left( \prod_{j = 1}^x a_j \right)^k\\
& = \sum_{T = 1}^n \sum_{d \mid T} d \mu (d)^2 T^{k x} \mu \left(\frac{T}{d} \right) \sum^{n / T}_{\{ a_x \} = 1} \left( \prod_{j = 1}^x a_j \right)^k\\
& = \sum_{T = 1}^n T^{k x} \left( \sum^{n / T}_{\{ a_x \} = 1} \left( \prod_{j = 1}^x a_j \right)^k \right) \sum_{d \mid T} d \mu (d)^2 \mu \left( \frac{T}{d} \right)\\
& = \sum_{T = 1}^n T^{k x} \left( \sum^{n / T}_{i = 1} i^k \right)^x \sum_{d \mid T} d \mu (d)^2 \mu \left( \frac{T}{d} \right)
\end{aligned} \]
到这里已经差不多化完了。令
\[ G (x) = \sum^x_{i = 1} i^k, H (x) = \sum_{d \mid x} d \mu (d)^2 \mu \left( \frac{x}{d} \right) \]
其中 $G (x)$ 显然可以预处理,而 $H (x)$ 是积性函数的卷积,故也是积性函数,其中
\[ H (p^c) = \begin{cases}
p - 1,& c = 1\\
-p, & c = 2  \\
0, & c > 2
\end{cases} \]
可以线性筛得到。总之
\[ S (n) = \sum_{T = 1}^n T^{k x} H (T) G \left( \left\lfloor \frac{n}{T} \right\rfloor \right)^x \]
可以用整数分块。


\section{P5349 幂}

\paragraph{题目大意}

即求
\[ S = \sum_{n=0}^\infty f(n)r^n \]

\paragraph{分析}

换序
\[ S = \sum_{n = 0}^{\infty} \left( \sum_{i = 0}^m f_i n^i \right) r^n
= \sum_{i = 0}^m f_i \sum_{n = 0}^{\infty} n^i r^n \]
令
\[ S_i = \sum_{n = 0}^{\infty} n^i r^n \]
套路的逐项相减,主动的凑二项式
\[ (1 - r) S_i = 1 + \sum_{n = 1}^{\infty} n^i r^n - \sum_{n = 1}^{\infty} (n - 1)^i r^n
= 1 + \sum_{n = 0}^{\infty} r^{n + 1} \sum_{j = 0}^{i - 1} \binom{k}{j} n^j \]
交换求和顺序
\[ (1 - r) S_i = 1 + \sum_{j = 0}^{i - 1} \binom{k}{j} \sum_{n = 0}^{\infty}r^{n + 1} n^j
= 1 + \sum_{j = 0}^{i - 1} \binom{k}{j} r S_j \]
再凑成完整的二项式卷积
\[ \frac{S_i - 1}{r} = \sum_{j = 0}^i \binom{k}{j} S_j \]
我们设 $\{ S_i \}$ 的 EGF 为 $g(x)$,可以得到方程
\[ \frac{g(x) - 1}{r} = {\rm e}^x g(x) \]
解得
\[ g(x) = \frac{1}{1 - r {\rm e}^x} \]
因此最终多项式逆即可,EGF 和 OGF 的转化就是点乘阶乘。



\chapter{2021 年 10 月}

\section{HDU6955 Xor sum}

\paragraph{题目大意}

给定整数序列 $\{a_n\}$,寻找满足区间异或和大于等于 $k$ 的连续序列中最短的。

若有多个同样长度的答案,输出位置靠前的,若不存在输出 \verb`-1`。

\paragraph{分析}

看到异或最值就应该想到 01trie。做前缀异或和,这样我们的问题就变成找到最靠近的两个数,其异或大于等于 $k$。

因为要求位置最靠前的,所以我选择倒着遍历。当遍历到第 $i$ 个数 $a_i$ 时,字典树已经维护好此位置之后的数据,即每个值出现最左边的位置。

我们只需从高位到低位遍历第 $j$ 位,分类讨论:

- 若 $k[j] = 1$,则说明 $a_i[j] \oplus a_u[j]$ 只能为 $1$,在字典树中选择 $a_i[j] \oplus 1$ 深入(什么都不做)。
- 若 $k[j] = 0$,那么若 $a_i[j] \oplus a_u[j] = 1$ 则一定大于 $k$,记录字典树中 $a_i[j] \oplus 1$ 的值,然后在字典树中选择 $a_i[j]$ 深入。

因为我们是从前往后查的,因此同样长度中总是先查到考前的,汇总有

\section{NC11255B Sample Game}

\paragraph{题目大意}

给定生成 $1$ 到 $n$ 数字的概率分布,不断生成随机数 $x$ 直到 $x$ 不是已经生成过的最大的数停止,求生成次数平方的期望。

\paragraph{分析}


设期望的随机次数为 $f_x = E(x)$,我们需要计算的次数为
$g_x = E(x^2)$。

关于 $f_i$ 的 DP 是显然的,计算有

\[ f_x = \sum_{i = 1}^{x - 1} p_i + \sum_{i = x}^n p_i(1 + f_i)
= 1 + \sum_{i = x}^n p_i f_i \]

容易观察到

\[ f_{x + 1} =(1 - p_x) f_x \Rightarrow f_x = \prod_{i = x}^n \frac{1}{1 - p_x} \]

接下来需要一个套路

\[ E((x + 1)^2) = E(x^2) + 2 E(x) + 1 \]

类似的可以推得

\[ g_x = \sum_{i = 1}^{x - 1} p_i + \sum_{i = x}^n p_i(g_i + 2 f_i + 1)  = 1
   + \sum_{i = x}^n p_i g_i + 2 \sum_{i = x}^n p_i f_i = \sum_{i = x}^n p_i
   g_i + 2 f_x - 1 \]

最终答案即是

\[ ans = \sum_{i = 1}^n p_i(g_i + 2 f_i + 1)  = g_1 \]

至此,倒着递推已经可以线性求解了。但是我们还可以继续优化,逐项相减有

\[ g_{x + 1} - g_x = 2(f_{x + 1} - f_x) - p_x g_x = - 2 p_x f_x - p_x g_x \]

即

\[ \frac{g_x}{f_x} - \frac{g_{x + 1}}{f_{x + 1}} = \frac{2}{1 - p_x} - 2 \]

因此

\[ {\rm ans} = g_1 = f_1 \left( \frac{g_n}{f_n} - 2(n - 1) + 2 \sum_{i =
   1}^{n - 1} \frac{1}{1 - p_x} \right) = \left( \prod_{i = x}^n \frac{1}{1 -
   p_x} \right) \left( 1 + 2 \sum_{i = 1}^n \frac{1}{1 - p_x} - 2 n \right) \]

至此,我们可以 $O(n)$ 的解决问题。

我们经过了很长的化简过程才导出这一结果,实际上生成函数更为快捷。

设 $f(x)$ 是生成长度为 $i$ 的非递减序列的生成函数,即 $P(len > i)$,可以推出

\[ f(x) = \prod_{i = 1}^n \frac{1}{1 - p_i x} \]

而我们需要求

\[ \sum_{i = 1}^{\infty} (f_{i - 1} - f_i) i^2 = \sum_{i = 0}^{\infty} f_i (2 i + 1) = 2 f'(1) + f(1) \]

化简即可得到上式。


\section{P3052 Cows in a Skyscraper G}

\paragraph{题目大意}

有 $n$ 头牛坐电梯,重量分别为 $c_i$,电梯的最大限额是 $W$,求最少分多少次能够全部上去。

\paragraph{分析}

注意 $n \leqslant 18$,枚举全排列是不行的。考虑用状态压缩,将第 $j$ 头奶牛的选和不选用状态 $i$ 的第 $j$ 位数字表示。设 $f_i$ 表示状态为 $i$ 时最小乘电梯次数,$g_i$ 表示此状态下最新那个电梯已经有的重量。

- 首先令 \verb`cow = i << (j - 1)`,若 \verb`i & cow` 为 $1$ 则说明这个奶牛已经坐上电梯了,不计算。
- 当 $W - g_i \geqslant c_i$ 时,最后那个电梯坐的下这头牛。
- 当 $W - g_i < c_i$ 时,只能新开一个电梯。

还是代码更清晰

\section{P4059 找爸爸}

\paragraph{题目大意}

给定两串 DNA 序列,可以在其中任意插空格,然后逐位比较,当两位都是字母时查表得到相似度。

长度为 $k$ 的空格有额外相似度 $- A - B(k - 1)$。求两序列的最大相似度。

\paragraph{分析}

若不考虑空格的贡献,容易想到二维 DP,记 ${\rm DP}[i,j]$ 是 $A$ 串到位置 $i$ 同时 $B$ 串到位置 $j$ 时最大的相似度,有

\[ {\rm DP}[i, j] = \max\{ {\rm DP}[i - 1, j - 1] + D[i, j], {\rm DP}[i - 1, j], {\rm DP}[i, j - 1] \} \]

当我们考虑空格的贡献时,可以发现空格的额外贡献只与前面一位有关。若两个序列此位都是空格,则把去掉后相似度一定会增加,故只有三种可能:设 ${\rm DP}_0$ 是结尾没有空格;${\rm DP}_1$ 是空格在 $A$ 串;${\rm DP}_2$ 是空格在 $B$ 串。

思考最后一个空格的转移方式,自然有方程

\[ \begin{aligned}
{\rm DP}_0[i, j] &= \max\{ {\rm DP}_0[i - 1, j - 1], {\rm DP}_1[i - 1, j - 1], {\rm DP}_2[i - 1, j - 1] \} + D[i, j] \\
{\rm DP}_1[i, j] &= \max\{ {\rm DP}_0[i, j - 1] - A, {\rm DP}_1[i, j - 1]- B, {\rm DP}_2[i, j - 1] - A\} \\
{\rm DP}_2[i, j] &= \max\{ {\rm DP}_0[i - 1, j] - A, {\rm DP}_1[i - 1, j] - A, {\rm DP}_2[i, j - 1] - B \} \\
\end{aligned} \]

随手加滚动数组 WA 了好久,发现每次都要清空为 \verb`-INF`,否则会 WA。

\section{P5171 Earthquake}

\paragraph{题目大意}

给定 $a,b,c$,求满足方程 $ax + by \leqslant c$ 的非负整数解的个数。

\paragraph{分析}

令 $n = \lfloor c / a \rfloor$,容易推出让我们求的是


\[ n + 1 + \sum_{x = 0}^{n} \left\lfloor \frac{c - ax}{b} \right\rfloor \]

直接套类欧是不行的,因为系数有负数。其他题解感觉推麻烦了,直接考虑水平翻转,令 $x \to (n -  x)$,代入有

\[ n + 1 + \sum_{x = 0}^{n} \left\lfloor \frac{ax + c - an}{b} \right\rfloor \]

显然 $c - an$ 是 \verb`c % a`,于是套类欧即可

\section{P5179 Fraction}

\paragraph{题目大意}

求最简分数 $p/q$ 满足

\[ \frac{a}{b} < \frac{p}{q} < \frac{c}{d} \]

若有多组解,输出 $q$ 最小的;仍有多组解,输出 $p$ 最小的一组。

\paragraph{分析}

分类讨论:

- 首先当 $\lfloor a/b \rfloor + 1 \leqslant \lceil c/d \rceil - 1$ 时,说明两个数之间存在一个整数,直接返回 $p = 1, q = \lfloor a / b \rfloor + 1$ 即可。
- 其次当 $a = 0$ 时,直接解得 $p = 1, q = \lfloor d / c \rfloor + 1$。
- 当 $a < b$ 时,这意味着我们无法直接求解。考虑翻转,转化问题为
\[ \frac{d}{c} < \frac{q}{p} < \frac{b}{a} \]
递归即可。
- 当 $a > b$ 时,考虑求解
\[ \frac{a}{b} - \left\lfloor\frac{a}{b}\right\rfloor < \frac{p}{q} - \left\lfloor\frac{a}{b}\right\rfloor < \frac{c}{d} - \left\lfloor\frac{a}{b}\right\rfloor \]
也是递归即可,还原出结果以后返回。

唯一需要确认的就是,翻转后仍是最优解,考虑反证。

设存在 $p_0 \geqslant p$ 且 $q_0 < q$ 使得 $p_0/q_0$ 在翻转前不是最优解,但是翻转后是最优解。从而有
\[ \frac{d}{c} < \frac{p_0}{q_0} < \frac{p_0}{q} \leqslant \frac{p}{q} < \frac{b}{a} \]
即 $p_0/q$ 是翻转前的更优解,与 $p/q$ 是最优解矛盾。


\chapter{2021 年 11 月}

\section{P1393 Mivik 的标题}

\paragraph{题目大意}

在字符集中有 $m$ 个不同的字符,求给定字符串 $S$ 在长度为 $n$ 的字符串的全集中出现的概率。

\paragraph{分析}

参考 [Mivik 的字符串公开赛](https://mivik.gitee.io/2020/solution/mivik-string-contest-title/)。

定义 $A$ 类字符串为,在整个字符串中 $S$ 只出现过一次,且恰好 $S$ 出现在末尾的字符串。

设 $\{f_i\}$ 是长度为 $i$ 的字符串中 $A$ 类字符串的个数。再定义 $\{g_i\}$ 是长度为 $i$ 的字符串中没有出现过 $S$ 的个数。

在长为 $i$ 的没有出现过 $S$ 的字符串后面加上一个字符,则可能出现 $S$ 也可能没有出现 $S$。
\[ m g_i = f_{i+1} + g_{i+1} \]
定义关于 $S$ 的数列 $\{s_i\}$ 为
\[ s_i = [i \ \text{is a period of}\ S] \]
即当 $i$ 是 $S$ 的一个循环节时,$s_i$ 为 $1$,否则为 $0$。

我们在所有未出现过 $S$ 的长为 $n - |S|$ 的字符串后面拼接上 $S$,当 $i$ 是 $S$ 的周期时 $n-i$ 是其完整出现 $S$ 的位置。有
\[ g_{n - |S|} = \sum_{i=0}^n f_{n - i} s_{i} \]
写成生成函数形式,可以得到 $f_i$ 的生成函数
\[ f(x) = \frac{x^{|S|}}{x^{|S|} + (1 - m x) s(x)} \]
我们要计算的是长为 $n$ 的符合要求的字符串总数,即是把字符串填充至 $n$ 的长度即可,即
\[ E(x) = \frac{f(x)}{1 - m x} \]
我们只需计算 $E[n]/m^n$ 即可。循环节可以用 KMP 预处理。


\end{document}