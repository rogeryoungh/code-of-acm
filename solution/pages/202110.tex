\chapter{2021 年 10 月}

\section{HDU6955 Xor sum}

\paragraph{题目大意}

给定整数序列 $\{a_n\}$,寻找满足区间异或和大于等于 $k$ 的连续序列中最短的。

若有多个同样长度的答案,输出位置靠前的,若不存在输出 \verb`-1`。

\paragraph{分析}

看到异或最值就应该想到 01trie。做前缀异或和,这样我们的问题就变成找到最靠近的两个数,其异或大于等于 $k$。

因为要求位置最靠前的,所以我选择倒着遍历。当遍历到第 $i$ 个数 $a_i$ 时,字典树已经维护好此位置之后的数据,即每个值出现最左边的位置。

我们只需从高位到低位遍历第 $j$ 位,分类讨论:

- 若 $k[j] = 1$,则说明 $a_i[j] \oplus a_u[j]$ 只能为 $1$,在字典树中选择 $a_i[j] \oplus 1$ 深入(什么都不做)。
- 若 $k[j] = 0$,那么若 $a_i[j] \oplus a_u[j] = 1$ 则一定大于 $k$,记录字典树中 $a_i[j] \oplus 1$ 的值,然后在字典树中选择 $a_i[j]$ 深入。

因为我们是从前往后查的,因此同样长度中总是先查到考前的,汇总有

\section{NC11255B Sample Game}

\paragraph{题目大意}

给定生成 $1$ 到 $n$ 数字的概率分布,不断生成随机数 $x$ 直到 $x$ 不是已经生成过的最大的数停止,求生成次数平方的期望。

\paragraph{分析}


设期望的随机次数为 $f_x = E(x)$,我们需要计算的次数为
$g_x = E(x^2)$。

关于 $f_i$ 的 DP 是显然的,计算有

\[ f_x = \sum_{i = 1}^{x - 1} p_i + \sum_{i = x}^n p_i(1 + f_i)
= 1 + \sum_{i = x}^n p_i f_i \]

容易观察到

\[ f_{x + 1} =(1 - p_x) f_x \Rightarrow f_x = \prod_{i = x}^n \frac{1}{1 - p_x} \]

接下来需要一个套路

\[ E((x + 1)^2) = E(x^2) + 2 E(x) + 1 \]

类似的可以推得

\[ g_x = \sum_{i = 1}^{x - 1} p_i + \sum_{i = x}^n p_i(g_i + 2 f_i + 1)  = 1
   + \sum_{i = x}^n p_i g_i + 2 \sum_{i = x}^n p_i f_i = \sum_{i = x}^n p_i
   g_i + 2 f_x - 1 \]

最终答案即是

\[ ans = \sum_{i = 1}^n p_i(g_i + 2 f_i + 1)  = g_1 \]

至此,倒着递推已经可以线性求解了。但是我们还可以继续优化,逐项相减有

\[ g_{x + 1} - g_x = 2(f_{x + 1} - f_x) - p_x g_x = - 2 p_x f_x - p_x g_x \]

即

\[ \frac{g_x}{f_x} - \frac{g_{x + 1}}{f_{x + 1}} = \frac{2}{1 - p_x} - 2 \]

因此

\[ {\rm ans} = g_1 = f_1 \left( \frac{g_n}{f_n} - 2(n - 1) + 2 \sum_{i =
   1}^{n - 1} \frac{1}{1 - p_x} \right) = \left( \prod_{i = x}^n \frac{1}{1 -
   p_x} \right) \left( 1 + 2 \sum_{i = 1}^n \frac{1}{1 - p_x} - 2 n \right) \]

至此,我们可以 $O(n)$ 的解决问题。

我们经过了很长的化简过程才导出这一结果,实际上生成函数更为快捷。

设 $f(x)$ 是生成长度为 $i$ 的非递减序列的生成函数,即 $P(len > i)$,可以推出

\[ f(x) = \prod_{i = 1}^n \frac{1}{1 - p_i x} \]

而我们需要求

\[ \sum_{i = 1}^{\infty} (f_{i - 1} - f_i) i^2 = \sum_{i = 0}^{\infty} f_i (2 i + 1) = 2 f'(1) + f(1) \]

化简即可得到上式。


\section{P3052 Cows in a Skyscraper G}

\paragraph{题目大意}

有 $n$ 头牛坐电梯,重量分别为 $c_i$,电梯的最大限额是 $W$,求最少分多少次能够全部上去。

\paragraph{分析}

注意 $n \leqslant 18$,枚举全排列是不行的。考虑用状态压缩,将第 $j$ 头奶牛的选和不选用状态 $i$ 的第 $j$ 位数字表示。设 $f_i$ 表示状态为 $i$ 时最小乘电梯次数,$g_i$ 表示此状态下最新那个电梯已经有的重量。

- 首先令 \verb`cow = i << (j - 1)`,若 \verb`i & cow` 为 $1$ 则说明这个奶牛已经坐上电梯了,不计算。
- 当 $W - g_i \geqslant c_i$ 时,最后那个电梯坐的下这头牛。
- 当 $W - g_i < c_i$ 时,只能新开一个电梯。

还是代码更清晰

\section{P4059 找爸爸}

\paragraph{题目大意}

给定两串 DNA 序列,可以在其中任意插空格,然后逐位比较,当两位都是字母时查表得到相似度。

长度为 $k$ 的空格有额外相似度 $- A - B(k - 1)$。求两序列的最大相似度。

\paragraph{分析}

若不考虑空格的贡献,容易想到二维 DP,记 ${\rm DP}[i,j]$ 是 $A$ 串到位置 $i$ 同时 $B$ 串到位置 $j$ 时最大的相似度,有

\[ {\rm DP}[i, j] = \max\{ {\rm DP}[i - 1, j - 1] + D[i, j], {\rm DP}[i - 1, j], {\rm DP}[i, j - 1] \} \]

当我们考虑空格的贡献时,可以发现空格的额外贡献只与前面一位有关。若两个序列此位都是空格,则把去掉后相似度一定会增加,故只有三种可能:设 ${\rm DP}_0$ 是结尾没有空格;${\rm DP}_1$ 是空格在 $A$ 串;${\rm DP}_2$ 是空格在 $B$ 串。

思考最后一个空格的转移方式,自然有方程

\[ \begin{aligned}
{\rm DP}_0[i, j] &= \max\{ {\rm DP}_0[i - 1, j - 1], {\rm DP}_1[i - 1, j - 1], {\rm DP}_2[i - 1, j - 1] \} + D[i, j] \\
{\rm DP}_1[i, j] &= \max\{ {\rm DP}_0[i, j - 1] - A, {\rm DP}_1[i, j - 1]- B, {\rm DP}_2[i, j - 1] - A\} \\
{\rm DP}_2[i, j] &= \max\{ {\rm DP}_0[i - 1, j] - A, {\rm DP}_1[i - 1, j] - A, {\rm DP}_2[i, j - 1] - B \} \\
\end{aligned} \]

随手加滚动数组 WA 了好久,发现每次都要清空为 \verb`-INF`,否则会 WA。

\section{P5171 Earthquake}

\paragraph{题目大意}

给定 $a,b,c$,求满足方程 $ax + by \leqslant c$ 的非负整数解的个数。

\paragraph{分析}

令 $n = \lfloor c / a \rfloor$,容易推出让我们求的是


\[ n + 1 + \sum_{x = 0}^{n} \left\lfloor \frac{c - ax}{b} \right\rfloor \]

直接套类欧是不行的,因为系数有负数。其他题解感觉推麻烦了,直接考虑水平翻转,令 $x \to (n -  x)$,代入有

\[ n + 1 + \sum_{x = 0}^{n} \left\lfloor \frac{ax + c - an}{b} \right\rfloor \]

显然 $c - an$ 是 \verb`c % a`,于是套类欧即可

\section{P5179 Fraction}

\paragraph{题目大意}

求最简分数 $p/q$ 满足

\[ \frac{a}{b} < \frac{p}{q} < \frac{c}{d} \]

若有多组解,输出 $q$ 最小的;仍有多组解,输出 $p$ 最小的一组。

\paragraph{分析}

分类讨论:

- 首先当 $\lfloor a/b \rfloor + 1 \leqslant \lceil c/d \rceil - 1$ 时,说明两个数之间存在一个整数,直接返回 $p = 1, q = \lfloor a / b \rfloor + 1$ 即可。
- 其次当 $a = 0$ 时,直接解得 $p = 1, q = \lfloor d / c \rfloor + 1$。
- 当 $a < b$ 时,这意味着我们无法直接求解。考虑翻转,转化问题为
\[ \frac{d}{c} < \frac{q}{p} < \frac{b}{a} \]
递归即可。
- 当 $a > b$ 时,考虑求解
\[ \frac{a}{b} - \left\lfloor\frac{a}{b}\right\rfloor < \frac{p}{q} - \left\lfloor\frac{a}{b}\right\rfloor < \frac{c}{d} - \left\lfloor\frac{a}{b}\right\rfloor \]
也是递归即可,还原出结果以后返回。

唯一需要确认的就是,翻转后仍是最优解,考虑反证。

设存在 $p_0 \geqslant p$ 且 $q_0 < q$ 使得 $p_0/q_0$ 在翻转前不是最优解,但是翻转后是最优解。从而有
\[ \frac{d}{c} < \frac{p_0}{q_0} < \frac{p_0}{q} \leqslant \frac{p}{q} < \frac{b}{a} \]
即 $p_0/q$ 是翻转前的更优解,与 $p/q$ 是最优解矛盾。

