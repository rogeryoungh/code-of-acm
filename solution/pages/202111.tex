\chapter{2021 年 11 月}

\section{P1393 Mivik 的标题}

\paragraph{题目大意}

在字符集中有 $m$ 个不同的字符,求给定字符串 $S$ 在长度为 $n$ 的字符串的全集中出现的概率。

\paragraph{分析}

参考 [Mivik 的字符串公开赛](https://mivik.gitee.io/2020/solution/mivik-string-contest-title/)。

定义 $A$ 类字符串为,在整个字符串中 $S$ 只出现过一次,且恰好 $S$ 出现在末尾的字符串。

设 $\{f_i\}$ 是长度为 $i$ 的字符串中 $A$ 类字符串的个数。再定义 $\{g_i\}$ 是长度为 $i$ 的字符串中没有出现过 $S$ 的个数。

在长为 $i$ 的没有出现过 $S$ 的字符串后面加上一个字符,则可能出现 $S$ 也可能没有出现 $S$。
\[ m g_i = f_{i+1} + g_{i+1} \]
定义关于 $S$ 的数列 $\{s_i\}$ 为
\[ s_i = [i \ \text{is a period of}\ S] \]
即当 $i$ 是 $S$ 的一个循环节时,$s_i$ 为 $1$,否则为 $0$。

我们在所有未出现过 $S$ 的长为 $n - |S|$ 的字符串后面拼接上 $S$,当 $i$ 是 $S$ 的周期时 $n-i$ 是其完整出现 $S$ 的位置。有
\[ g_{n - |S|} = \sum_{i=0}^n f_{n - i} s_{i} \]
写成生成函数形式,可以得到 $f_i$ 的生成函数
\[ f(x) = \frac{x^{|S|}}{x^{|S|} + (1 - m x) s(x)} \]
我们要计算的是长为 $n$ 的符合要求的字符串总数,即是把字符串填充至 $n$ 的长度即可,即
\[ E(x) = \frac{f(x)}{1 - m x} \]
我们只需计算 $E[n]/m^n$ 即可。循环节可以用 KMP 预处理。
