\chapter{2021 年 5 月}

\section{P2522 Problem B}

\paragraph{题目大意}

即求
\[ \sum_{i=a}^b \sum_{j=c}^d [\gcd(i,j) = k] \]

\paragraph{分析}

容易想到,独立出函数 $f(k)$ 使得
\[ f(k) = \sum_{i=1}^x \sum_{j=1}^y [\gcd(i,j) = k] \]

利用 Mobius 反演化简,设 $F(d)$
\[ F(n) = \sum_{n \mid d} f(d) = \sum_{i=1}^x \sum_{j=1}^y [n \mid i][n \mid j] = \left\lfloor \frac{x}{n} \right\rfloor \left\lfloor \frac{y}{n} \right\rfloor \]

反演化简有
\[ f(n) = \sum_{n \mid d} \mu\left(\frac{d}{n}\right)F(d) = \sum_{t=1}^{\min(x,y)} \mu(t) \left\lfloor \frac{x}{tn} \right\rfloor \left\lfloor \frac{y}{tn} \right\rfloor \]

预处理出 $\mu(t)$ 的前缀和,剩下的就是一个二重分块了。

\paragraph{类似题目}

P2158 仪仗队:即 $k = 1$ 的特殊情况。

P3455 ZAP-Queries:几乎一样。

P2257 YY 的 GCD:比这题难,单独开篇。

\section{P2257 YY 的 GCD}

\paragraph{题目大意}
\[ \sum_{i=1}^N \sum_{j=1}^M [\gcd(i,j) \in \mathbb{P}] \]

\paragraph{分析}

先转化一下
\[ \sum_{i=1}^N \sum_{j=1}^M [\gcd(i,j) \in \mathbb{P}] = \sum_{p}\sum_{i=1}^N \sum_{j=1}^M [\gcd(i,j) = p] \]
在 P2522 中得到
\[ \sum_{i=1}^x \sum_{j=1}^y [\gcd(i,j) = k] = \sum_{t=1}^{\min(x,y)} \mu(t) \left\lfloor \frac{x}{tk} \right\rfloor \left\lfloor \frac{y}{tk} \right\rfloor \]
代入有
\[ \sum_{p} \sum_{t=1}^{\min(x,y)} \mu(t) \left\lfloor \frac{x}{tp} \right\rfloor \left\lfloor \frac{y}{tp} \right\rfloor \]
令 $T = tp$,$T$ 的上界应还是 $\min(x,y)$,代入有
\[ \sum_{T=1}^{\min(x,y)}  \left\lfloor \frac{x}{T} \right\rfloor \left\lfloor \frac{y}{T} \right\rfloor \sum_{p \mid T} \mu\left(\frac{T}{p}\right) \]
后面部分可以利用 Euler 筛预处理的。令
\[ dp(T) = \sum_{p \mid T} \mu\left(\frac{T}{p}\right) \]
$f_i$ 是其前缀和,最后整除分块即可。

\section{P2303 Longge 的问题}

\paragraph{题目大意}

即求
\[ \sum_{i=1}^n \gcd(i,n) \]

\paragraph{分析}

联想到
\[ \varphi(n) = \sum_{i=1}^n [\gcd(i,n) = 1] \]
尝试凑这个形式
\[ \begin{aligned}
\sum_{i=1}^n \gcd(i,n) &= \sum_{d \mid n} d \sum_{i=1}^n [\gcd(i,n) = d] \\
&= \sum_{d \mid n} d \sum_{i=1}^{n/d} [\gcd(i,n/d) = 1] \\
&= \sum_{d \mid n} d \varphi(n/d)
\end{aligned} \]
这里其实已经可以过题了,但还可以再瞎搞一下,令
\[ \sum_{d \mid n} d \varphi(n/d) = n\sum_{d \mid n} \frac{\varphi(d)}{d} = nf(n)\]
尝试证明 $f(n)$ 是一个积性函数。设 $\gcd(a,b) = 1$,有
\[ \begin{aligned}
f(a)f(b) &= \left(\sum_{d_1 \mid a} \frac{\varphi(d_1)}{d_1}\right) \left(\sum_{d_2 \mid b} \frac{\varphi(d_2)}{d_2}\right)\\
&= \sum_{d_1 \mid a} \sum_{d_2 \mid b} \frac{\varphi(d_1)}{d_1} \frac{\varphi(d_2)}{d_2}\\
&= \sum_{d_1 \mid a} \sum_{d_2 \mid b} \frac{\varphi(d_1d_2)}{d_1d_2}\\
&= f(ab)
\end{aligned} \]
再来推一下素数,注意 $1 \mid p^k$,有
\[ f(p^k) = \sum_{d \mid p^k} \frac{\varphi(d)}{d} = \sum_{i=0}^k \frac{\varphi(p^i)}{p^i} = k\left(1 - \frac{1}{p}\right) + 1 \]
类似于 $\varphi(m)$ 唯一分解形式,我们还有
\[ f(n) = \prod_{i=1}^sf(p_i^{k_i}) = \prod_{i=1}^s \frac{k_ip_i - k_i + p_i}{p_i} \]

\section{P2350 外星人}

\paragraph{题目大意}

设
\[ \varphi_{x+1}(m) = \varphi(\varphi_x(m)) \]
求最小的 $x$ 使得 $\varphi_x(m) = 1$。

\paragraph{分析}

注意到仅有 $\varphi(1) = \varphi(2) = 1$,再有公式
\[ \varphi\left(\prod_{i=1}^mp_i^{q_i}\right) = \prod_{i=1}^m(p_i-1)p_i^{q_i-1} \]
因此从唯一分解形式的角度来看,迭代一次消去了一个 $2$,也生成了一些 $2$。

考虑设 $f(n)$ 为 $\varphi(n)$ 中因子 $2$ 的个数。设 $\gcd(a,b) = 1$,可以证明 $f(ab) = f(a) f(b)$。同时 $f(p) = f(p-1)$。这说明 $f(x)$ 是一个积性函数,可以用 Euler 筛递推。

注意若没有质因子 $2$,则答案需要加 $1$。