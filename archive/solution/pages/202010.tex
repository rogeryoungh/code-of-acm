\chapter{2020 年 10 月}

\section{P1095 守望者的逃离}

\paragraph{题目大意}

守望者的可以在一秒逃出 $17 {\rm m}$,或者消耗 $10$ 点魔法值闪现 $60 {\rm m}$。原地休息时每秒回复 $4$ 点魔法值。

守望者开始有 $M$ 点魔法,需要在 $T$ 秒内逃离距离 $S$。若能逃离则求最短逃离时间,否则求最远距离。

\paragraph{分析}

设 $s_1$为一直走路,$s_2$ 为一直闪现恢复。当 $s_2$ 快了就把 $s_1$ 更新为 $s_2$。

\section{P1923 求第 $k$ 小的数}

\paragraph{题目大意}

给定数列,求第 $k$ 小的数。

\paragraph{分析}

考虑分治,随便选一个数 $x$,然后把比 $x$ 大的数移到右边,比 $x$ 小的数移到左边。因此得到了数 $x$ 的排位,若恰为第 $k$ 个则返回,否则根据大小在左右寻找。

该算法是不稳定的,期望复杂度 $O(n)$,最坏复杂度 $O(n^2)$。实际上存在最坏复杂度为 $O(n)$ 的 BFPTR 算法,但因为其常数过大实现复杂而很少应用。

\section{P1928 外星密码}

\paragraph{题目大意}

我们将重复的 $n$ 个字符串 \verb`S` 压缩为 \verb`[nS]`,且存在多重压缩。给定一个压缩结果,展开它。

\paragraph{分析}

考虑用类似状态机的方式解析字符串。当读到正常字符时,把它添到 $s$ 末尾;读到左括号时,则递归 \verb`read()`,重复 $n$ 遍添到 $s$ 末尾;右括号或文本结束则返回 $s$。

\section{P1990 覆盖墙壁}

\paragraph{题目大意}

有 \verb`I` 形和 \verb`L` 形两种砖头,分别能覆盖 2 个单元和 3 个单元。求 $2 \times n$ 的墙有多少不重复的覆盖方式,结果对 $10^4$ 取模。

\paragraph{分析}

其中 \verb`I` 形砖块仅有横放和竖放两种。关键在于 \verb`L` 形,两个 \verb`L` 形之间可以用 \verb`I` 形填充,这让情况变得复杂起来。

对于 $F_n$ 的递推,我们可以想到:在 $F_{n-1}$ 后放一个 \verb`I` 形砖块;在 $F_{n-2}$ 后放两个横着的 \verb`I` 形砖块。对于更前面的递推,较为复杂。

两个 \verb`L` 形砖块对齐,上下翻转也可以,即 $2 F_{n-3}$;两个 \verb`L` 形砖块可以对顶放,空缺恰用一个 \verb`I` 填充,即 $2 F_{n-4}$;类似 $2F_{n-3}$,中间可以再插入两个 \verb`I` 形,即 $2 F_{n-5}$……直到 \verb`I` 形和 \verb`L` 形砖块恰好铺满墙壁,即 $2F_{0}$。

容易得到我们的递推式
\[ F_n = F_{n-1} + F_{n-2} + 2 \sum_{i=0}^{n-3} F_i \]
利用错位相减法,不难化简得到
\[ F_n = 2 F_{n-1} + F_{n-3} \]

\section{P1090 合并果子}

\paragraph{题目大意}

可以合并两堆果子成一堆新果子,消耗两堆果子数目之和的体力。给定 $n$ 堆果子的数目 $a_i$,求体力耗费最小的方案。

\paragraph{分析}

很容易猜到贪心结论,不断选取两个最小的堆进行合并。本质上是证明 Huffman 树的构造的正确性,有点复杂。

\section{P4995 跳跳!}

\paragraph{题目大意}

给定石头的高度 $h$,从第 $i$ 块石头跳到第 $j$ 块上耗费体力 $(h_i-h_j)^2$ ,求最耗体力的路径。

\paragraph{分析}

容易猜到贪心结论,是不断的在剩余石头中最大最小的来回跳。

考虑证明结论,设 $h_i$ 是将要跳的序列,展开求和式
\[ S = \sum_{k=1}^{n-1}(h_k-h_{k+1})^2 = \sum_{k=1}^nh_k^2 - 2\sum_{k=1}^{n-1}h_kh_{k+1} \]
注意到平方和为一个定值,重点在后半式。记 $H_k = h_{k+1}$,有
\[ \sum_{k=1}^{n-1}h_kH_k \]
利用高中时学的排序不等式,有
\[ \text{反序和} \leqslant \text{乱序和} \leqslant \text{顺序和} \]
于是有反序最小。

\section{P1077 摆花}

\paragraph{题目大意}

要从 $n$ 种花中挑出 $m$ 盆展览,其中第 $i$ 种花不得多于 $a_i$ 种。求有几种选法。

\paragraph{分析}

考虑动态规划,记状态 $dp[i,j]$ 为摆完前 $i$ 种花,共 $j$ 盆时的方案数。容易得到递推式
\[ dp[i,j] = \sum_{k = j - \min(a_i, j)}^j dp[i-1,k] \]
边界条件是 $dp[0,0] = 1$。可以用滚动数组、前缀和优化。
